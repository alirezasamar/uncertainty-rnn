\documentclass{utmthesis}
\usepackage{graphicx}
\usepackage{url} 
\usepackage[pages=some]{background}
\usepackage{algorithm,algorithmic}
\usepackage{algpseudocode}
\usepackage{amsmath}
\usepackage{cite}


\begin{document}

% Required information
\title{Uncertainty in Recurrent Neural Networks}
\author{Alireza Samar}
\degree{Master of Philosophy}
\specialization{Machine Learning}
\intakeyear{2016}
\faculty{Advanced Informatics School}
\titledate{April 2017}
\award{4}
% Options for Award 
% 1. Bachelor Degree Project Report
% 2. Master's Project Report (By course work)
% 3. Master's Dissertation (By course work and research)
% 4. Master's Thesis (By research)
% 5. Doctor of Philosophy Thesis
% 6. Other PhD Thesis
% 7. Generic PhD Thesis
% 8. Thesis Proposal
\superone{Dr. Siti Sophiayati Yuhaniz}
%\supertwo{M.Y. Other Supervisor}
%\superthree{Third SV}
%\superfour{Fourth SV}
%\superfive{Fifth SV}

% Option for two-page printing
\newgeometry{top=2.5cm,left=4cm,right=2.5cm,bottom=2.5cm,twoside}

% Option to add watermark page
% Comment for final version	
\backgroundsetup{scale=1,angle=0,opacity=.1,hshift=0.25in,vshift=-0.5in,contents={\includegraphics[width=5cm]{figs/utm02.jpg}}}
% \watermarkpage

% Mandatory pages
%\coverpage
%\superpage
%\certification
%\frontmatter
\maketitle
%\declaration

%\begin{dedication}
%Dedication\
%\end{dedication}

%\begin{acknowledgement}
%Acknowledgement
%\end{acknowledgement}

\begin{abstract}
Recurrent neural networks (RNN), and especially Deep RNNs have outperformed in various fields from computer vision, and language processing to physics, biology, and manufacturing. This means the deep or multi-layer architecture of neural networks are being extensively used in these fields; for instance convolutional neural networks (CNN) as image processing tools, and recurrent neural networks (RNN) as sequence processing model various from language modeling to image captioning.

In traditional sciences fields such as physics and biology, analysis of model uncertainty is crucial. Nowadays, the control of critical systems is being handed to machine learning-based systems that could affect our day-to-day life directly, but the question is "what should these systems do with high uncertainty in output?" As a solution, by having model confidence, uncertain outputs can be treated as the special case by a human to make the decision.

This work aims to propose a novel thechnique and develop tools to measure uncertainty estimates by adoption of Bayes by Backprop to the given network, especially in deep recurrent neural networks. The performance of proposed technique in this work will be extensively evaluated with widely studied benchmark.

\end{abstract}

%\begin{abstrak}
%Ini adalah abstrak Bahasa Melayu
%\end{abstrak}

\tableofcontents
\listoftables
\listoffigures

%List of abbreviation 
\listofabbre
\addabbre{ANN}{Artificial Neural Network}
\addabbre{NN}{Neural Network}
\addabbre{BNN}{Bayesian Neural Network}
\addabbre{RNN}{Recurrent Neural Network}
\addabbre{BBB}{Bayes by Backprop}
\addabbre{CNN}{Convolutional Neural Network}
\addabbre{KL}{Kullback-Leibler}
\addabbre{GP}{Gaussian Process}
\addabbre{MC}{Monte Carlo}
\addabbre{VI}{Variational Inference}
\addabbre{MCMC}{Markov Chain Monte Carlo}
\addabbre{LSTM}{Long Short-Term Memory}
\addabbre{SRT}{Stochastic Regularisation Technique (such as dropout)}
\addabbre{i.e.}{Id est (“it is”)}
\addabbre{w.r.t.}{With respect to}

%List of symbols 
\listofsymbols
\addsymbol{$A$}{A rolled NN}
\addsymbol{\textbf{A}}{Matrix}
\addsymbol{a}{vector}
\addsymbol{$t$}{Steps/Length of RNN}
\addsymbol{$x_t$}{Input value}
\addsymbol{$\eta$}{Trained parameter}
\addsymbol{$c$}{Internal core sate}
\addsymbol{$h$}{Exposed sate}
\addsymbol{$i_t$}{Input gate}
\addsymbol{$f_t$}{Forget gate}
\addsymbol{$h_t$}{Exposed sate}
\addsymbol{$W$}{Weights (biases)}

%List of appendices
%\listofappendices

\onehalfspacing
\mainmatter

\chapter{Introduction}
\label{chap:intro}
\section{Problem Background}
Introduction to the thesis \cite{b2} to the thesis \cite{okamoto2004improved}. This section attempts to give a brief introduction to quantum computing. Before entering the microscopic world of quantum computing, we revisit the present digital system commonly used by the masses.  The current digital system is based on binary digits, commonly known as bits.  Each bit is represented with a binary value called ``logic 0'' or ``logic 1'' and the number of distinct states is $2^n$, where $n$ is the number of bits.  Physically, these logic values are typically represented by two different voltage levels. In this thesis, such computers are referred to as a \emph{classical computer}.
\section{State-of-the-Arts}
\section{Problem Statement}
\section{Objective and Scope}
\section{Organization}
\chapter{Literature Review}
\label{chap:lit.review}

\section{Introduction}

\subsection{Recurrent Neural Networks}

Recurrent Neural Networks (RNNs) are in forefront of recent development and advances in \textit{deep learning} by making able neural networks to deal with sequences data, which is a major shortcoming in ANN. If the data is based on sequence of events in a video or text, the traditional neural network can't do reasoning for a single event based on its previous one. To tackle this issue RNNs have loops which enables them to persist the information.

\begin{figure}[p]
	\centering
	\includegraphics[scale=0.4]{./figs/rnn-rolled}
	\caption[A Rolled Recurrent Neural Networks]{Recurrent Neural Networks (RNNs) uses loops.}
	\label{fig:rnn-rolled}
\end{figure}

As it shown in \textbf{Figure \ref{fig:rnn-rolled}}, a selected neural network, $A$ takes the input $x_t$ and outputs the value of $h_t$. this might not show how data goes from one step to the next one in a same network until you unroll the loop and see chain architecture of recurrent neural networks that makes them the best choice for sequential data, \textbf{Figure \ref{fig:rnn-unrolled}}.

\begin{figure}[p]
	\centering
	\includegraphics[scale=0.4]{./figs/rnn-unrolled}
	\caption[An Unrolled Recurrent Neural Networks]{An Unrolled Recurrent Neural Networks (RNNs).}
	\label{fig:rnn-unrolled}
\end{figure}

While RNNs is being used in variety of applications from language modeling to image captioning, the essential to all these achievement is the RNN-LSTMs \cite{Hochreiter1997}. An enhanced version of RNNs that outperforms better than the standard RNN.

\subsection{Bayes by Backprop}

Bayes by Backprop \cite{Blundell2015a} is a variational inference scheme for learning the posterior distribution on the weights of a neural network.
The posterior distribution on parameters of the network $\theta \in \mathbb{R}^d$, $q(\theta)$ is typically taken to be a Gaussian with mean parameter $\mu\in \mathbb{R}^d$ and standard deviation parameter $\sigma\in \mathbb{R}^d$, denoted $\mathcal{N}(\theta|\mu,\sigma)$ and it's a diagonal covariance matrix. Where $d$ is the dimensionality of the parameters of the network (usually refers to scale of millions).
Let $\log p(y|\theta, x)$ be the log-likelihood of the neural network, then the network is trained by minimising the variational free energy, where $p(\theta)$ is a prior on the parameters:

\begin{align}
	\label{eq:elbo}
	\mathcal{L}(\theta) &=
	\mathbb{E}_{q(\theta)}\left[\log \frac{q(\theta)}{p(y|\theta, x)p(\theta)}\right],
\end{align}

The \textbf{algorithm \ref{alg:bbb}} shows the Bayes by Backprop Monte Carlo procedure for minimising the above equation with respect to the mean and standard deviation parameters of the posterior $q(\theta)$.

Minimising the variational free energy equation is equivalent to maximising the log-likelihood $\log p(y|\theta, x)$ subject to a KL complexity term in the parameters of the network that acts as a regulariser:

\begin{align}
	\label{eq:klelbo}
	\mathcal{L}(\theta) &=
	- \mathbb{E}_{q(\theta)}\left[\log p(y|\theta, x) \right]
	+ \kl{q(\theta)}{p(\theta)}.
\end{align}

Gaussian process in a case with zero mean prior, the KL term acts as a form of weight decay on the mean parameters, where the rate of weight decay is automatically tuned by the standard deviation parameters of the prior and posterior.

\begin{algorithm}[ht]
	\caption{Bayes by Backprop}
	\label{alg:bbb}
	\begin{algorithmic}
		\STATE{Sample $\epsilon \sim \mathcal{N}(0, I)$, $\epsilon \in \mathbb{R}^d$.}
		\STATE{Set network parameters to $\theta = \mu + \sigma\epsilon$.}
		\STATE{Do forward propagation and backpropagation as normal.}
		\STATE{Let $g$ be the gradient with respect	\label{eq:elbo}
			\mathcal{L}(\theta) &=
			\mathbb{E}_{q(\theta)}\left[\log \frac{q(\theta)}{p(y|\theta, x)p(\theta)}\right], to $\theta$ from backpropagation.} 
		\STATE{Let $g^{KL}_\theta, g^{KL}_\mu, g^{KL}_\sigma$ be the gradients of $\log \mathcal{N}(\theta|\mu, \sigma) - \log p(\theta)$ with respect to $\theta$, $\mu$ and 
			$\sigma$ respectively.} 
		\STATE{Update $\mu$ according to the gradient $g + g^{KL}_\theta + g^{KL}_\mu$.} 
		\STATE{Update $\sigma$ according to the gradient $(g + g^{KL}_\theta) \epsilon + g^{KL}_\sigma$.}
	\end{algorithmic}
\end{algorithm}

The uncertainty afforded by Bayes by Backprop trained networks has been used successfully for training feed-forward NN models for supervised learning and also to help exploration to reinforcement learning agents \cite{Blundell2015a}, \cite{Lipton2016}, \cite{Houthooft2016}, but until now, BBB has not been applied to recurrent neural networks.

\section{Backprop Through Time}
\label{sec:bptt}

As illustrated in \ref{fig:rnn-rolled}, the core of an recurrent neural networks (RNNs), $f$, is a neural network that maps the RNN state at step $t$, $s_t$ and an input observation $x_t$ to a new RNN state $s_{t+1}$, $f: (s_t, x_t) \mapsto s_{t+1}$.

For comparison, an LSTM-powered RNN core \cite{Hochreiter1997} has a state $s_t = (c_t, h_t)$ where $c$ is an internal core state and $h$ is the exposed state. Intermediate gates modulate the effect of the inputs on the outputs, gates like the input gate $i_t$, forget gate $f_t$ and output gate $o_t$. The relationship between the inputs, outputs and internal gates of an LSTM cell can be explained as:

\begin{align*}
i_t &= \sigma(W_i [x_t, h_{t-1}]^T + b_i), \\
f_t &= \sigma(W_f [x_t, h_{t-1}]^T + b_f), \\
c_t &= f_t c_{t-1} + i_t \tanh(W_c [x_t, h_{t-1}] + b_c), \\
o_t &= \sigma(W_o [x_t, h_{t-1}]^T + b_o), \\
h_t &= o_t \tanh(c_t),
\end{align*}

In the above statement, $W_i$ ($b_i$), $W_f$ ($b_f$), $W_c$ ($b_c$) and $W_o$ ($b_o$) are the weights (biases) that are affecting the input gate, forget gate, cell update, and output gate accordingly.

An RNN can be trained on a sequence of $T$ using backpropagation through time where the RNN is unrolled $T$ times like a feed-forward network.
Which can be achieve with forming the feed-forward network with inputs
$x_1, x_2, \dots, x_T$ and initial state $s_0$:

\begin{align}
s_1 &= f(s_0, x_1), \nonumber \\
s_2 &= f(s_1, x_2), \nonumber \\
&\dots \nonumber \\
\label{eq:unroll}
s_T &= f(s_{T-1}, x_T), 
\end{align}

where $s_T$ is the total length (final state) of the RNN.
Referring to the unrolled RNN for $T$ steps as in \eqref{eq:unroll} by $s_{1:T} = F_T(x_{1:T}, s_0)$,
where $x_{1:T}$ is the sequence of input vectors and $s_{1:T}$ is the sequence of corresponding states. However, the truncated version of the algorithm can be seen as taking $s_0$ as the last state of the previous batch, $s_T$.

This RNN parameters are trained in the same way as a feed-forward neural network and a loss is applied to the states $s_{1:T}$ of the RNN, and then backpropagation is being used to update the weights of the trained network.
Since the weights in each of the unrolled step are shared, each weight of the RNN core receives $T$ gradient contributions when the it is unrolled for $T$ steps.


\subsection{Model Confidence}

\subsection{Model Uncertainty and Safety}

\section{State-of-the-Arts}

\section{Limitations}
\begin{enumerate}
\item Mentor~Graphics 2
\begin{enumerate}
\item item 3
\end{enumerate}
\item item 4
\end{enumerate}

\section{Research Gaps}
The processing at layer-5%

\chapter{Research Methodology}
\label{chap:method}
\section{Top-level View}
\section{Research Activities}
\section{Controllables vs. Obseravables}
\section{Techniques}
\section{Tools and Platforms}
\section{Chapter Summary}
\chapter{Analysis and Design}
\label{chap:proposed.work}

\section{Bayes by Backprop}

Bayes by Backprop \cite{Blundell2015a} is a variational inference scheme for learning the posterior distribution on the weights of a neural network.
The posterior distribution on parameters of the network $\theta \in \mathbb{R}^d$, $q(\theta)$ is typically taken to be a Gaussian with mean parameter $\mu\in \mathbb{R}^d$ and standard deviation parameter $\sigma\in \mathbb{R}^d$, denoted $\mathcal{N}(\theta|\mu,\sigma)$ and it's a diagonal covariance matrix. Where $d$ is the dimensionality of the parameters of the network (usually refers to scale of millions).
Let $\log p(y|\theta, x)$ be the log-likelihood of the neural network, then the network is trained by minimising the variational free energy, where $p(\theta)$ is a prior on the parameters:

\begin{align}
	\label{eq:elbo}
	\mathcal{L}(\theta) &=
	\mathbb{E}_{q(\theta)}\left[\log \frac{q(\theta)}{p(y|\theta, x)p(\theta)}\right],
\end{align}

The \textbf{algorithm \ref{alg:bbb}} shows the Bayes by Backprop Monte Carlo procedure for minimising the above equation with respect to the mean and standard deviation parameters of the posterior $q(\theta)$.

Minimising the variational free energy equation is equivalent to maximising the log-likelihood $\log p(y|\theta, x)$ subject to a KL complexity term in the parameters of the network that acts as a regulariser:

\begin{align}
	\label{eq:klelbo}
	\mathcal{L}(\theta) &=
	- \mathbb{E}_{q(\theta)}\left[\log p(y|\theta, x) \right]
	+ \kl{q(\theta)}{p(\theta)}.
\end{align}

Gaussian process in a case with zero mean prior, the KL term acts as a form of weight decay on the mean parameters, where the rate of weight decay is automatically tuned by the standard deviation parameters of the prior and posterior.

\begin{algorithm}[ht]
	\caption{Bayes by Backprop}
	\label{alg:bbb}
	\begin{algorithmic}
		\STATE{Sample $\epsilon \sim \mathcal{N}(0, I)$, $\epsilon \in \mathbb{R}^d$.}
		\STATE{Set network parameters to $\theta = \mu + \sigma\epsilon$.}
		\STATE{Do forward propagation and backpropagation as normal.}
		\STATE{Let $g$ be the gradient with respect	\label{eq:elbo}
			\mathcal{L}(\theta) &=
			\mathbb{E}_{q(\theta)}\left[\log \frac{q(\theta)}{p(y|\theta, x)p(\theta)}\right], to $\theta$ from backpropagation.} 
		\STATE{Let $g^{KL}_\theta, g^{KL}_\mu, g^{KL}_\sigma$ be the gradients of $\log \mathcal{N}(\theta|\mu, \sigma) - \log p(\theta)$ with respect to $\theta$, $\mu$ and 
			$\sigma$ respectively.} 
		\STATE{Update $\mu$ according to the gradient $g + g^{KL}_\theta + g^{KL}_\mu$.} 
		\STATE{Update $\sigma$ according to the gradient $(g + g^{KL}_\theta) \epsilon + g^{KL}_\sigma$.}
	\end{algorithmic}
\end{algorithm}

The uncertainty afforded by Bayes by Backprop trained networks has been used successfully for training feed-forward NN models for supervised learning and also to help exploration to reinforcement learning agents \cite{Blundell2015a}, \cite{Lipton2016}, \cite{Houthooft2016}, but until now, BBB has not been applied to recurrent neural networks.

\section{Adoption of Bayes by Backprop to RNNs}
\label{sec:tbbbtt}

Applying Bayes by Backprop (BBB) to RNNs is illustrated in \textbf{Figure \ref{fig:lstmbbb}} where the weight matrices of the RNN are drawn from a distribution (learnt by BBB).
However, this direct application raises two questions: when to sample the parameters of the RNN, and how to weight the contribution of the KL regulariser of \eqref{eq:klelbo}.
To address these concerns in the adaptation of BBB to RNNs, Algorithm \ref{alg:rnnbbb} given where:

\begin{figure}
	\centering
	\includegraphics[width=\linewidth]{figs/LSTMBBB}
	\caption{Adoption of Bayes by Backprop (BBB) to an RNN.}
	\label{fig:lstmbbb}
\end{figure}

The variational free energy of \eqref{eq:klelbo} for an RNN on a sequence of length $T$ is:

\begin{align}
	\mathcal{L}(\theta) &=
	- \mathbb{E}_{q(\theta)}\left[\log p(y_{1:T}|\theta, x_{1:T}) \right]
	\nonumber \\
	&\phantom{=}
	+ \kl{q(\theta)}{p(\theta)},
	\label{eq:rnnelbo}
\end{align}

where $p(y_{1:T}|\theta, x_{1:T})$ is the likelihood of a sequence produced when the states of an unrolled RNN $F_T$ are fed into an appropriate probability distribution.
The parameters of the entire network are $\theta$.
Although the RNN is unrolled $T$ times, each weight is penalised just once by the KL term, rather than $T$ times.
Also clear from \eqref{eq:rnnelbo} is that when a Monte Carlo approximation is taken to the expectation, the parameters $\theta$ should be held fixed throughout the entire sequence.

Two complications arise to the above naive derivation in practice: firstly, sequences are often long enough and models sufficiently large, that unrolling the RNN for the whole sequence is prohibitive.
Secondly, to reduce variance in the gradients, more than one sequence is trained at a time.
Thus the typical regime for training RNNs involves training on mini-batches of truncated sequences.

Let $B$ be the number of mini-batches and $C$ the number of truncated sequences (cuts),
then \eqref{eq:rnnelbo} as:
\begin{align}
	\mathcal{L}(\theta) &=
	- \mathbb{E}_{q(\theta)}\left[\log \prod_{b=1}^B \prod_{c=1}^{C} p(y^{(b,c)}|\theta, x^{(b,c)}) \right]
	\nonumber \\
	&\phantom{=}
	+ \kl{q(\theta)}{p(\theta)},
\end{align}
where the $(b,c)$ superscript denotes elements of $c$th truncated sequence in the $b$th minibatch.
Thus the free energy of mini-batch $b$ of a truncated sequence $c$ can be written as:
\begin{align}
	\mathcal{L}_{(b,c)}(\theta) &=
	- \mathbb{E}_{q(\theta)}\left[\log p(y^{(b,c)}|\theta, x^{(b,c)}, s^{(b,c)}_\text{prev}) \right]
	\nonumber \\
	&\phantom{=}
	+ w^{(b,c)}_\text{KL} \kl{q(\theta)}{p(\theta)},
	\label{eq:weightelbo}
\end{align}
where $w^{(b,c)}_\text{KL}$ distributes the responsibility of the KL cost among minibatches and truncated sequences (thus $\sum_{b=1}^B \sum_{c=1}^C w^{(b,c)}_\text{KL} = 1$), and $s^{(b,c)}_\text{prev}$ refers to the initial state of the RNN for the minibatch $x^{(b,c)}$.
In practice, by choosing $w^{(b,c)}_\text{KL} = \frac{1}{C B}$ so that the KL penalty is equally distributed among all mini-batches and truncated sequences.
The truncated sequences in each subsequent mini-batches are picked in order, and so $s^{(b,c)}_\text{prev}$ is set to the last state of the RNN for $x^{(b,c-1)}$.

Finally, the question of when to sample weights follows naturally from taking a Monte Carlo approximations to \eqref{eq:weightelbo}: for each minibatch, sample a fresh set of parameters.

\begin{algorithm}[ht]
	\caption{Bayes by Backprop for RNNs}
	\label{alg:rnnbbb}
	\begin{algorithmic}
		\STATE{Sample $\epsilon \sim \mathcal{N}(0,I)$, $\epsilon \in \mathbb{R}^d$.}
		\STATE{Set network parameters to $\theta = \mu + \sigma\epsilon$.}
		\STATE{Sample a minibatch of truncated sequences $(x,y)$.}
		\STATE{Do forward propagation and backpropagation as normal on minibatch.}
		\STATE{Let $g$ be the gradient with respect to $\theta$ from backpropagation.}
		\STATE{Let $g^{KL}_\theta, g^{KL}_\mu, g^{KL}_\sigma$ be the gradients of $\log \mathcal{N}(\theta|\mu, \sigma) - \log p(\theta)$ w.r.t. $\theta$, $\mu$ and $\sigma$ respectively.}
		\STATE{Update $\mu$ according to the gradient $\frac{g + \frac{1}{C}g^{KL}_\theta}{B} + \frac{g^{KL}_\mu}{B C}$.}
		\STATE{Update $\sigma$ according to the gradient $\left(\frac{g + \frac{1}{C} g^{KL}_\theta}{B}\right) \epsilon + \frac{g^{KL}_\sigma}{B C}$.}
	\end{algorithmic}
\end{algorithm}
%\include{Chapters/ch5}
%\chapter{Conclusion}
\label{chap:conclusion}
\section{Research Outcomes}
\section{Contributions to Knowledge}
\section{Future Works}

\bibliographystyle{utmthesis-numbering}
\bibliography{reference}

%\appendix
%\chapter{Do not use long titles.}
%\chapter{Pseudo-codes}
%\chapter{Time-series Results}

%This is required to make List of Appendices possible. Remove when have no appendix.
%\endmatter
\end{document}
